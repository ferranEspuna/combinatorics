\documentclass{amsart}
\renewcommand{\baselinestretch}{1.1}

\parskip 2mm

\usepackage{amsthm}
\usepackage{amsmath}
\usepackage{amsfonts}
\usepackage{amssymb}
\usepackage{fullpage}
\usepackage{mathtools}

\newtheorem{theirtheorem}{Theorem}
\newtheorem{theirproposition}{Proposition}
\renewcommand{\thetheirtheorem}{\Alph{theirtheorem}}
\renewcommand{\thetheirproposition}{\Alph{theirproposition}}


\theoremstyle{plain}
\newtheorem{theorem}{\textbf{Theorem}}[section]
\newtheorem{lemma}[theorem]{\textbf{Lemma}}
\newtheorem{corollary}[theorem]{\textbf{Corollary}}
\newtheorem{proposition}[theorem]{\textbf{Proposition}}
\newtheorem{claim}{\textbf{Claim}}
\newtheorem{conjecture}{\textbf{Conjecture}}[section]

\theoremstyle{definition}
\newtheorem{rk}{\textbf{Remark}}

\newcommand{\Summ}[1]{\underset{#1}{\sum}}
\newcommand{\sti}[2]{\left\{\begin{array}{c} #1 \\ #2 \end{array}\right\}}

\newcommand{\diam}{\emph{diam}}
\newcommand{\conv}{\mbox{Conv}}
\newcommand{\C}{\mathcal {C}}
\newcommand{\R}{\mathbb{R}}
\newcommand{\Z}{\mathbb{Z}}
\newcommand{\N}{\mathbb{N}}
\newcommand{\F}{\mathbb{F}}

\newcommand{\B}{\mathcal{B}}
\newcommand{\A}{\mathcal{A}}
\newcommand{\G}{\mathcal{G}}
\newcommand{\D}{\mathcal{D}}

\newcommand{\ov}[1]{\overline{#1}}

\newcommand{\nn}{\nonumber}

\def\st{2}

\thispagestyle{empty}

\begin{document}

    {\Large Combinatorics -- MAMME}
    {\Large Chapter 2.1 -- Ramsey Theorem}

    \vspace{0.5cm}

    \hrule

    \vspace{0.5cm}

    \begin{enumerate}


    \item[\textbf{Problem 5:}] Let $(G, \cdot)$ be a finite \emph{semigroup} (that is, we only require the operation to be associative).
    Prove that there is $a \in G$ such that $a^2 = a$.
    [Hint: Use Ramsey's Theorem on triangles in a sufficiently large sequence of elements of $G$.]
    \end{enumerate}

    \paragraph{\textbf{Solution (by Ferran Espuña):}} Pick any $x \in G$.
    We will finitely color the sets of two positive integers as follows:

    \begin{align*}
        c: \binom{\mathbb{Z}_+}{2} & \longrightarrow G \\
        \{i < j\} & \longmapsto x^{j-i}
    \end{align*}

    \noindent By Ramsey's Theorem, there is an infinite set $A \subset \mathbb{Z}_+$ such that $c$ is constant on $\binom{A}{2}$.
    In particular, for any $i < j < k$ in $A$, we have $x^{j-i} = x^{k-i} = x^{k-j} \eqqcolon a$,
    with $a^2 = x^{j-i} \cdot x^{k-j} = x^{k-i} = a$.

    \begin{rk}
        The fact that the operation is associative is crucial for the proof, if somewhat hidden in it.
        Indeed, if the operation were not associative, we would need to define exponentiation of elements of $G$ more carefully
        (e.g.\ by choosing to repeatedly apply the operation from the left or from the right).
        In that case, we could not just add the exponents of a product as we do in the last step of the proof.

    \end{rk}

\end{document}

