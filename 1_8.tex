\documentclass{amsart}
\renewcommand{\baselinestretch}{1.1}

\parskip 2mm

\usepackage{amsthm}
\usepackage{amsmath}
\usepackage{amsfonts}
\usepackage{amssymb}
\usepackage{fullpage}

\newtheorem{theirtheorem}{Theorem}
\newtheorem{theirproposition}{Proposition}
\renewcommand{\thetheirtheorem}{\Alph{theirtheorem}}
\renewcommand{\thetheirproposition}{\Alph{theirproposition}}


\theoremstyle{plain}
\newtheorem{theorem}{\textbf{Theorem}}[section]
\newtheorem{lemma}[theorem]{\textbf{Lemma}}
\newtheorem{corollary}[theorem]{\textbf{Corollary}}
\newtheorem{proposition}[theorem]{\textbf{Proposition}}
\newtheorem{claim}{\textbf{Claim}}
\newtheorem{conjecture}{\textbf{Conjecture}}[section]

\theoremstyle{definition}
\newtheorem{rk}{\textbf{Remark}}

\newcommand{\Summ}[1]{\underset{#1}{\sum}}
\newcommand{\sti}[2]{\left\{\begin{array}{c} #1 \\ #2 \end{array}\right\}}

\newcommand{\diam}{\emph{diam}}
\newcommand{\conv}{\mbox{Conv}}
\newcommand{\C}{\mathcal {C}}
\newcommand{\R}{\mathbb{R}}
\newcommand{\Z}{\mathbb{Z}}
\newcommand{\N}{\mathbb{N}}
\newcommand{\F}{\mathbb{F}}

\newcommand{\B}{\mathcal{B}}
\newcommand{\A}{\mathcal{A}}
\newcommand{\G}{\mathcal{G}}
\newcommand{\D}{\mathcal{D}}

\newcommand{\ov}[1]{\overline{#1}}

\newcommand{\nn}{\nonumber}

\def\st{2}

\thispagestyle{empty}

\begin{document}

    {\Large Combinatorics -- MAMME}
    {\Large Chapter 1.1 -- Sperner's Theorem}

    \vspace{0.5cm}

    \hrule

    \vspace{0.5cm}

    \begin{enumerate}


    \item[\textbf{Problem 8:}] Let $P$ be a ranked poset with the LYM property.
    A \emph{regular covering of chains} of $P$ is a family $\mathcal{C}$ of maximal chains such that
    any two elements of $P$ with the same rank are in the same number of chains in $\mathcal{C}$.
    Prove that, for any real valued function $f : P \longrightarrow \mathbb{R}$ and every subset $X \subset P$,

    \begin{equation}\label{eq:whatwewant}
    \sum_{x \in X} \frac{f(x)}{  \lvert P_{r(x)} \rvert } \leq \max_{c \in \mathcal{C}} \sum_{x \in X \cap c} f(x)
    \end{equation}
    Deduce the following theorem of Erd\H{o}s: If $\mathcal{A}$ is a family of subsets of $[n]$ such that
    the longest chain in $\mathcal{A}$ has length at most $k$, then

    \begin{equation}\label{eq:erdos}
    \lvert \mathcal{A} \rvert \leq \sum_{i = 0}^{k-1}\binom{n}{\lfloor (n+i)/2 \rfloor}
    \end{equation}

    \noindent [Hint: For the first inequality consider the function F: $\mathcal{C} \to \R $ definded as $F(c) = \sum_{x \in c} f(x)$ and use double counting.
    For the second inequality, chose $f(x) = \lvert P_{r(x)} \rvert$.]
    \end{enumerate}

    \paragraph{\textbf{Solution (by Ferran Espuña):}} Every element of $P_k$
    is in the same number of chains of $\mathcal{C}$ for all $k$.
    Conversely, every
    chain in $\mathcal{C}$ is maximal, so it contains one element of each $P_k$.
    Therefore, every element $x \in P_{r(x)}$
    is contained in exactly $\frac{\lvert \mathcal{C} \rvert}{\lvert P_{r(x)} \rvert}$ chains of $\mathcal{C}$.
    Therefore,

    \begin{equation}\label{eq:intermediate_1}
    \sum_{c \in \mathcal{C}} \sum_{x \in X \cap c} f(x) =
    \sum_{x \in X} \frac{\lvert \mathcal{C} \rvert f(x)}{\lvert P_{r(x)} \rvert}
    = \lvert \mathcal{C} \rvert \sum_{x \in X} \frac{f(x)}{\lvert P_{r(x)} \rvert}
    \end{equation}

    However,

    \begin{equation}\label{eq:intermediate_2}
    \sum_{c \in \mathcal{C}} \sum_{x \in X \cap c} f(x) \leq \lvert \mathcal{C} \rvert \max_{c \in \mathcal{C}} \sum_{x \in X \cap c} f(x)
    \end{equation}
    Putting these two equations together, and dividing by $\lvert \mathcal{C} \rvert$, we get~\eqref{eq:whatwewant}.

    \noindent To get Erd\H{o}s' theorem, we take $P$ to be the poset of subsets of $[n]$ ordered by inclusion.
    To get a regular covering of $P$, we just take $\mathcal{C}$ to be the family of all maximal chains.
    This is a regular covering because every element of $P_k$ is in exactly $k!(n-k)!$ chains.
    We will apply~\eqref{eq:whatwewant} to the set $X = \mathcal{A}$.
    We define $f(x) = \lvert P_{r(x)} \rvert$, so that the left hand side of~\eqref{eq:whatwewant} is just
    $\sum_{x \in X} 1 = \lvert \mathcal{A} \rvert$.
    Because $\lvert P_j \rvert = \binom{n}{j}$,
    and chains don't have elements of the same rank, the right hand side of~\eqref{eq:whatwewant} is
    the sum of at most $k$ binomial coefficients $\binom{n}{j}$ for \emph{different} $j$.
    Taking the $k$ largest ones (and recalling that $\binom{n}{j} = \binom{n}{n-j}$), we get Erd\H{o}s' theorem.

    \begin{rk}
        \eqref{eq:erdos} is a generalization of Sperner's theorem,
        which we can recover by taking $k=1$ so that $\mathcal{A}$ is an antichain.
    \end{rk}

    \begin{rk}
    The LYM property wasn't used in the proof, so~\eqref{eq:whatwewant}
    holds for any ranked poset with a regular covering.
    In fact, we can recover the LYM property by repeating the same argument with $f(x) \equiv 1 $
    and letting $\mathcal{A}$ be an antichain so that the right hand side of~\eqref{eq:whatwewant} is 1.
    let us state this formally:

    \begin{proposition}
        Let $P$ be a ranked poset.
        If there exists a regular covering of $P$, then $P$ has the LYM property.
    \end{proposition}

    \end{rk}
\end{document}

