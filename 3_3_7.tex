\documentclass{amsart}
\renewcommand{\baselinestretch}{1.1}

\parskip 2mm

\usepackage{amsthm}
\usepackage{amsmath}
\usepackage{amsfonts}
\usepackage{amssymb}
\usepackage{fullpage}
\usepackage{mathtools}

\newtheorem{theirtheorem}{Theorem}
\newtheorem{theirproposition}{Proposition}
\renewcommand{\thetheirtheorem}{\Alph{theirtheorem}}
\renewcommand{\thetheirproposition}{\Alph{theirproposition}}


\theoremstyle{plain}
\newtheorem{theorem}{\textbf{Theorem}}[section]
\newtheorem{lemma}[theorem]{\textbf{Lemma}}
\newtheorem{corollary}[theorem]{\textbf{Corollary}}
\newtheorem{proposition}[theorem]{\textbf{Proposition}}
\newtheorem{claim}{\textbf{Claim}}
\newtheorem{conjecture}{\textbf{Conjecture}}[section]

\theoremstyle{definition}
\newtheorem{rk}{\textbf{Remark}}

\newcommand{\Summ}[1]{\underset{#1}{\sum}}
\newcommand{\sti}[2]{\left\{\begin{array}{c} #1 \\ #2 \end{array}\right\}}

\newcommand{\diam}{\emph{diam}}
\newcommand{\conv}{\mbox{Conv}}
\newcommand{\C}{\mathcal {C}}
\newcommand{\R}{\mathbb{R}}
\newcommand{\Z}{\mathbb{Z}}
\newcommand{\N}{\mathbb{N}}
\newcommand{\F}{\mathbb{F}}

\newcommand{\B}{\mathcal{B}}
\newcommand{\A}{\mathcal{A}}
\newcommand{\G}{\mathcal{G}}
\newcommand{\D}{\mathcal{D}}

\newcommand{\ov}[1]{\overline{#1}}

\newcommand{\nn}{\nonumber}

\def\st{2}

\thispagestyle{empty}

\begin{document}

    {\Large Combinatorics -- MAMME}
    {\Large Chapter 3.3 -- Lovász Local Lemma}

    \vspace{0.5cm}

    \hrule

    \vspace{0.5cm}

    \begin{enumerate}


    \item[\textbf{Problem 7:}]
    A family $\mathcal{A}$ of $m$-subsets of [n] is a \emph{$k$-covering} if every
    $i \in [n]$ is contained in at least $k$ sets of $\mathcal{A}$.
    A $k$-covering $\mathcal{A}$ is \emph{decomposable} if there exists a partition
    $\mathcal{A} = \mathcal{A}_1 \sqcup \mathcal{A}_2$ such that $\mathcal{A}_1$ and $\mathcal{A}_2$
    are both $1$-coverings.
    Show that if each point is in at most $t \leq 2^{k-4}/m$ sets then $\mathcal{A}$ is decomposable.



    \end{enumerate}

    \paragraph{\textbf{Solution (by Ferran Espuña):}}
    We will work with a random partition of $\mathcal{A}$ into $\mathcal{A}_1$ and $\mathcal{A}_2$.
    That is, each set of $\mathcal{A}$ is independently assigned to $\mathcal{A}_1$ with probability $1/2$,
    and to $\mathcal{A}_2$ with probability $1/2$.
    For convenience, we will define
    \[
        U_i \coloneqq \bigcup_{S \in \mathcal{A}_i}S
    \]
    Our goal will be to prove that the event
    \[
        B \coloneqq \left\{ U_1 = U_2 = [n]\right\}
    \]
    occurs with positive probability.
    For this, we will define the events
    \[
        B_i \coloneqq \left\{ i \notin U_1 \text{ or } i \notin U_2 \right\}, \quad i \in [n]
    \]
    and note that \[
        B = \cap_{i \in [n]} \overline{B_i}
    \]
    We will see that these events satisfy the conditions of the Lovász Local Lemma and we will be done.
    First of all, because each $i \in [n]$ is in $c_i \geq k$ sets of $\mathcal{A}$,
    the probability that all of them end up in the same set is
    \[
        2^{-c_i} + 2^{-c_i} = 2^{1-c_i} \leq 2^{1-k} \eqqcolon p
    \]
    where each summand corresponds to all sets ending up in $\mathcal{A}_1$ or $\mathcal{A}_2$.
    Next, note that if we define \[
        D_i \coloneqq \left\{
        i \neq j \in [n] \mid
        \{i, j\} \subseteq S \text{ for some } S \in \mathcal{A}
        \right\}
    \]
    then $B_i$ is independent of all $B_j$ with $j \notin D_i$.
    Let us now calculate a bound for all $|D_i|$.
    For this, we will double count the number of pairs in \[
        P_i \coloneqq \left\{ (S, j) \mid i \neq j \in [n], \{i, j\} \subset S \in \mathcal{A} \right\}
    \]
    On the one hand,
    if we start counting on the first coordinate this is clearly at most $(m - 1)t$.
    On the other hand, it is at least $|D_i|$ (one for each possible $j$). Therefore, \[
        |D_i| \leq (m - 1)t \eqqcolon d
    \]
    All together,\[
        ep(d+1) = e2^{1-k}((m-1)t + 1) \leq e2^{1-k}mt \leq e2^{1-k}2^{k-4} = e/8 < 1
    \]
    \qed

\end{document}
